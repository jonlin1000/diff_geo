\subsection{Motivation of the definition of a Topology}

Tamvakis describes topology as the theory of \underline{convergence}. Now what exactly does this mean?

Consider a sequence $\{x_n\}$ of real numbers.

We say that $\{x_n\}$ converges to $x$ if and only if for all $\varepsilon > 0$ there exists a natural number $N$ such that $|x_n - x| < \ve$ for all $n > N$. Then, we can say that $f: \Real \to \Real$ is continuous if $x_n \to x \implies f(x_n) \to f(x)$.

Now, how do we generalize this? For instance, let $X$ be a set. Can we define convergence of sequences in $X$? We need a notion of \textbf{distance} for this to work! But what if there is no natural distance function? For example, define 
\[X = \{f: \Real \to \Real~\text{where }f\text{ is continuous}\}.\] 
For this set, we have a notion of convergence (the canonical convergence of functions) but no notion of a distance function, or \textit{metric}.

In general for an interesting set $X$ a metric that is meaningful is very hard to define.

Another related question about topology is related to maps and the distance that they preserve. This is related to the question about associating spaces within various equivalence classes.

Consider a map $f:(X, d) \to (Y, d)$ between two metric spaces. We can define \textit{isometries} as the maps $f$ which preserve distance, that is, $d(f(x), f(y)) = d(x, y)$. It is not too difficult to show that isometries are also bijections. We can associate $X$ and $Y$ as the ``same'' space. But for many questions that one might want to answer restricting the metric is often too restrictive for questions that people want to answer.

Suppose that we relax the notion of isometries, where we suppose that instead of preserving distance, we only have that $f$ is continuous with a continuous inverse. A cube may be deformed into more interesting shapes and spaces, such as a sphere. It's an interesting note that we can't continuously deform a cube into a torus (or doughnut).

\begin{question}
How do we define continuity without the notion of distance? These issues lead into the definition of what a topology is.
\end{question}

\subsection{Background on Sets}

The following definition introduces some notation:
\begin{definition}
We say $a \in A$ to mean that the object $a$ is an element of a set $A$. The symbol $\varnothing$ represents the empty set. We denote the union of two sets as $A \cap B$ and the intersection of two sets as $A \cup B$. We define the difference of two sets 
\[A - B = A \\ B = \{a \in A | a \not\in B\}.\]
If a superset $X$ is understood we write $A^c$ or $\overline{A}$ to mean $X - A$.
\end{definition}

\begin{proposition}
For any three sets $A$, $B$, and $C$ we have that $A \cap (B \cup C) = (A \cap B) \cup (A \cap C)$ and $A \cup (B \cap C) = (A \cup B) \cap (A \cup C)$.
\end{proposition}

\begin{proposition}[DeMorgan's Laws]
For any two sets $A$ and $B$ we have that $(A \cap B)^c = A^c \cup B^c$ and $(A \cup B)^c = A^c \cap B^c$. 
\end{proposition}

Given an index set $I \neq \varnothing$, we can consider a family of sets $\{A_i\}_{i \in I}$, one for each $i$ in $I$. We can then form arbitrary unions and intersections of sets. For indexes like these DeMorgan's laws still hold.

There are various ways of constructing new sets.

\begin{definition}
The powerset of $A$ $\mathcal{P}(A)$ is defined as 
\[\mathcal{P}(A) = \{I | I \subset A\}.\]
The Cartesian product of $X$ and $Y$, $X \times Y$ is defined as 
\[X \times Y = \{(x, y) | x \in X, y \in Y\}.\]
\end{definition}

Later we will define a Cartesian product over an arbitrary index set.

Given non-empty sets $A$ and $B$ we define $f:A \to B$ as a rule of assignment which assigns to every $a \in A$ some unique $b \in B$. We can describe the map in the following manners: $f(a) = b$ or $a \mapsto b$. More formally, a function $f: A \to B$ is identified with its graph $\Gamma(f) \subset A \times B$. $\Gamma(f)$ has the condition that if $(a, b) \in \Gamma(f)$ and $(a, c) \in Gamma(f)$ then $b = c$.

Given a map $f: A \to B$ we call $A$ the domain and $B$ the codomain. The \textit{image} of $f$ is the set 
\[f(A) = \{f(a) | a \in A\}.\]
We can restrict functions to subsets in the obvious way. We can also define composition, one-to-one (injective), onto (surjective), and bijective functions fairly easily.

\begin{definition}
If $f:A \to B$ is a bijection we have a function $f^{-1}:B \to A$ called the \textit{inverse function}. We have the composition rules $(f^{-1} \circ f)(a) = a$ and $(f \circ f^{-1})(b) = b$ for all $a$ and $b$.
\end{definition}

\begin{definition}
Given a function $f: A \to B$ and a subset $B_0 \subset B$ we define 
\[f^{-1}(B_0) = \{a \in A | f(a) \in B_0\}.\]

This technically means that $f^{-1}$ is a function from the powerset of $B$ to the powerset of $A$. An interesting thing to note is that the inverse image preserves both unions and intersections in the canonical sense.
\end{definition}

\begin{definition}
A relation $R$ on $A$ is a subset $R \subset A \times A$. We write $xRy$ if $(x, y) \in R$. 
\end{definition}

Most interesting relations have restrictions on the relation. The most important types of relations are \textbf{equivalence relations} and \textbf{order relations}.

\begin{definition}
Denote the relation $R$ by $\sim$. $\sim$ is an equivalence relation if it is reflexive, symmetric, and transitive. One can pick up an abstract algebra book to learn what these terms mean.
\end{definition}

An important property of equivalence relations is that we can define equivalence classes for each $x \in A$, defined by $C_x = \{y \in A \mid x \sim y\}.$ 

\begin{proposition}
The following are equivalent:
\begin{itemize}
	\item $C_x \cap C_y = \varnothing$
	\item $C_x = C_y$
	\item $x \sim y$.
\end{itemize}
\end{proposition}
\begin{proof}
	Suppose $z \in C_x \cap C_y$. Then $z \sim x$ and $z \sim y$, implying $x \sim y$. Then, if $y \sim z$ then $x \sim y$ implies $x \sim z$ by transitivity. So $z \in C_y \implies z \in C_x$. The converse is proved similarly. If $C_x = C_y$ then their intersection is just $C_x$ and by definition we have that $C_x$ is non-empty ($x \sim x$).
\end{proof}

Thus any two equivalence classes are disjoint or equal.

\begin{definition}
	A \textit{partition} of a set $A$ is a collection $\mathcal{D}$ of disjoint non empty subsets of $A$ whose union is $A$.
\end{definition}

So a collection of all equivalence classes form a partition of $A$. Conversely, given a partition of $A$, we can create an equivalence relation $x \sim y \iff x \in P_i, y \in P_i$ where $P_i$ is one of the disjoint subsets. We denote the set of all equivalence clss of an equivalence relation as $A/\sim$. Note the similarity to quotienting, and how the process is basically the same but with more generality.

\begin{definition}[Munkres]
A partial order of a set $A$ is a relation $\leq$ on $A$ which is reflexive, anti-symmetric ($x \leq y, y \leq x \implies x = y$) and transitive. A partial order is called a total order if every pair of two elements of $A$ are comparable.
\end{definition}
From now on, when we refer to ordered sets we are referring to totally ordered sets.
If $(A, \leq)$ is an ordered set, and $a < b$, then we define the open interval $(a, b) = \{x \in A\mid a < x < b\}$ where $<$ means the obvious non-equality thing. If $(a, b) = \varnothing$ then $b$ is called the immediate successor of $a$ and $a$ is called the immediate predecessor of $b$. We can similarly define other types of intervals, in fact it's done in Munkres chapter 2.

\begin{definition}
Suppose $(A, \leq_1)$ and $(B, \leq_2)$ are two ordered sets. Then we define the dictionary order relation on $A \times B$ by $(a, b) < (a', b')$ if $a <_1 a'$ or $a = a'$ and $b<_2 b$.
\end{definition}

\begin{definition}
If $A$ and $B$ are ordered sets we say that $A$ and $B$ have the same order type or are order isomorphic if there is a bijection $f: A \to B$ which is order preserving: $a < a' \implies f(a) <_2 f(a')$.
\end{definition}

$\mathbb{N}$ is not order isomorphic to $\mathbb{Z}$: note that $\mathbb{N}$ has a least element. Similarly, $\mathbb{Z}$ is not order isomorphic to $\mathbb{Q}$ because integers have immediate successors and predecessors and no rational number has that property.

\begin{definition}
	Let $(A, \leq)$ be an ordered set. $b \in A$ is a largest element of $A$ if $b \geq a$ for all $a \in A$. $b$ is an upper bound for $X$ if $b \geq x$ for all $x \in X$. We say that $b$ is a \textbf{least upper bound} if $b$ is an upper bound but for any upper bound $b'$ for x we have $b \leq b'$.
	Least elements, lower bounds, and greatest lower bounds can be defined analogously.
\end{definition}

Finally, Tamvakis discussed some axioms about ``infinity'', which he did not believe in. (He is a finitist.) 
\begin{definition}
	The \textbf{Axiom of Infinity} essentially asserts the existence of $\mathbb{N}$.
\end{definition}

Related to infinity, suppose that $\{A_i\}_{i \in I}$ is a family of sets, with $A_i neq \varnothing$ for all $i \in I$. We want to define the product 
\[\prod_{i \in I}A_i.\]
If I is finite, then WLOG $I = \{1, \dotsc, n\}$ and we can just define the product as the Cartesian Product. How do we define an infinite product? The idea is to think of the tuple $(a_1, \dots, a_n)$ as a function with values $a(1), \dotsc, a(n)$ where $a_i \in A_i$ for all $i$. 

\begin{definition}
	The product 
	\[\prod_{i \in I}A_i\] is the set of all functions \[a: I \to \bigcup_{i \in I}A_i\] such that $a(i) \in A_i$ for all $i \in I$. 
\end{definition}

The famous \textbf{Axiom of Choice} asserts that this product is non-empty. Such a function $a$ is called a \textit{choice function} for the family of sets.