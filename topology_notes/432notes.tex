\documentclass[12pt]{article}
\usepackage[margin=1in]{geometry}
\usepackage{amsmath}
\usepackage{amssymb}
\usepackage{amsthm}

\theoremstyle{plain}
\newtheorem{theorem}{Theorem}
\newtheorem{corollary}{Corollary}
\newtheorem{proposition}{Proposition}

\theoremstyle{definition}
\newtheorem{definition}{Definition}
\newtheorem{question}{Question}
\newtheorem{example}{Example}

\newcommand{\ve}{\varepsilon}
\newcommand{\Real}{\mathbb{R}}

\title{MATH432 Fall 19 Notes}
\author{Jonathan Lin}
\date{\today}

\begin{document}
\maketitle

\section{Some Logistics}
Homework is put on the professor's website \verb|math.umd.edu/~harryt|. It's the most important part of the class. Homework is assigned on Thursday and is due the next Thursday. The midterms are October 3rd and November 14th, and the final is December 14. Office hours are 1-2 on Tuesday and Wednesday in room 4419.

\section{Lecture 8/27: Motivation and review of Set Theory}

\subsection{Motivation of the definition of a Topology}

Tamvakis describes topology as the theory of \underline{convergence}. Now what exactly does this mean?

Consider a sequence $\{x_n\}$ of real numbers.

We say that $\{x_n\}$ converges to $x$ if and only if for all $\varepsilon > 0$ there exists a natural number $N$ such that $|x_n - x| < \ve$ for all $n > N$. Then, we can say that $f: \Real \to \Real$ is continuous if $x_n \to x \implies f(x_n) \to f(x)$.

Now, how do we generalize this? For instance, let $X$ be a set. Can we define convergence of sequences in $X$? We need a notion of \textbf{distance} for this to work! But what if there is no natural distance function? For example, define 
\[X = \{f: \Real \to \Real~\text{where }f\text{ is continuous}\}.\] 
For this set, we have a notion of convergence (the canonical convergence of functions) but no notion of a distance function, or \textit{metric}.

In general for an interesting set $X$ a metric that is meaningful is very hard to define.

Another related question about topology is related to maps and the distance that they preserve. This is related to the question about associating spaces within various equivalence classes.

Consider a map $f:(X, d) \to (Y, d)$ between two metric spaces. We can define \textit{isometries} as the maps $f$ which preserve distance, that is, $d(f(x), f(y)) = d(x, y)$. It is not too difficult to show that isometries are also bijections. We can associate $X$ and $Y$ as the ``same'' space. But for many questions that one might want to answer restricting the metric is often too restrictive for questions that people want to answer.

Suppose that we relax the notion of isometries, where we suppose that instead of preserving distance, we only have that $f$ is continuous with a continuous inverse. A cube may be deformed into more interesting shapes and spaces, such as a sphere. It's an interesting note that we can't continuously deform a cube into a torus (or doughnut).

\begin{question}
How do we define continuity without the notion of distance? These issues lead into the definition of what a topology is.
\end{question}

\subsection{Background on Sets}

The following definition introduces some notation:
\begin{definition}
We say $a \in A$ to mean that the object $a$ is an element of a set $A$. The symbol $\varnothing$ represents the empty set. We denote the union of two sets as $A \cap B$ and the intersection of two sets as $A \cup B$. We define the difference of two sets 
\[A - B = A \\ B = \{a \in A | a \not\in B\}.\]
If a superset $X$ is understood we write $A^c$ or $\overline{A}$ to mean $X - A$.
\end{definition}

\begin{proposition}
For any three sets $A$, $B$, and $C$ we have that $A \cap (B \cup C) = (A \cap B) \cup (A \cap C)$ and $A \cup (B \cap C) = (A \cup B) \cap (A \cup C)$.
\end{proposition}

\begin{proposition}[DeMorgan's Laws]
For any two sets $A$ and $B$ we have that $(A \cap B)^c = A^c \cup B^c$ and $(A \cup B)^c = A^c \cap B^c$. 
\end{proposition}

Given an index set $I \neq \varnothing$, we can consider a family of sets $\{A_i\}_{i \in I}$, one for each $i$ in $I$. We can then form arbitrary unions and intersections of sets. For indexes like these DeMorgan's laws still hold.

There are various ways of constructing new sets.

\begin{definition}
The powerset of $A$ $\mathcal{P}(A)$ is defined as 
\[\mathcal{P}(A) = \{I | I \subset A\}.\]
The Cartesian product of $X$ and $Y$, $X \times Y$ is defined as 
\[X \times Y = \{(x, y) | x \in X, y \in Y\}.\]
\end{definition}

Later we will define a Cartesian product over an arbitrary index set.

Given non-empty sets $A$ and $B$ we define $f:A \to B$ as a rule of assignment which assigns to every $a \in A$ some unique $b \in B$. We can describe the map in the following manners: $f(a) = b$ or $a \mapsto b$. More formally, a function $f: A \to B$ is identified with its graph $\Gamma(f) \subset A \times B$. $\Gamma(f)$ has the condition that if $(a, b) \in \Gamma(f)$ and $(a, c) \in Gamma(f)$ then $b = c$.

Given a map $f: A \to B$ we call $A$ the domain and $B$ the codomain. The \textit{image} of $f$ is the set 
\[f(A) = \{f(a) | a \in A\}.\]
We can restrict functions to subsets in the obvious way. We can also define composition, one-to-one (injective), onto (surjective), and bijective functions fairly easily.

\begin{definition}
If $f:A \to B$ is a bijection we have a function $f^{-1}:B \to A$ called the \textit{inverse function}. We have the composition rules $(f^{-1} \circ f)(a) = a$ and $(f \circ f^{-1})(b) = b$ for all $a$ and $b$.
\end{definition}

\begin{definition}
Given a function $f: A \to B$ and a subset $B_0 \subset B$ we define 
\[f^{-1}(B_0) = \{a \in A | f(a) \in B_0\}.\]

This technically means that $f^{-1}$ is a function from the powerset of $B$ to the powerset of $A$. An interesting thing to note is that the inverse image preserves both unions and intersections in the canonical sense.
\end{definition}

\begin{definition}
A relation $R$ on $A$ is a subset $R \subset A \times A$. We write $xRy$ if $(x, y) \in R$. 
\end{definition}

Most interesting relations have restrictions on the relation. The most important types of relations are \textbf{equivalence relations} and \textbf{order relations}.

\begin{definition}
Denote the relation $R$ by $\sim$. $\sim$ is an equivalence relation if it is reflexive, symmetric, and transitive. One can pick up an abstract algebra book to learn what these terms mean.
\end{definition}

An important property of equivalence relations is that we can define equivalence classes for each $x \in A$, defined by $C_x = \{y \in A \mid x \sim y\}.$ 

\begin{proposition}
The following are equivalent:
\begin{itemize}
	\item $C_x \cap C_y = \varnothing$
	\item $C_x = C_y$
	\item $x \sim y$.
\end{itemize}
\end{proposition}
\begin{proof}
	Suppose $z \in C_x \cap C_y$. Then $z \sim x$ and $z \sim y$, implying $x \sim y$. Then, if $y \sim z$ then $x \sim y$ implies $x \sim z$ by transitivity. So $z \in C_y \implies z \in C_x$. The converse is proved similarly. If $C_x = C_y$ then their intersection is just $C_x$ and by definition we have that $C_x$ is non-empty ($x \sim x$).
\end{proof}

Thus any two equivalence classes are disjoint or equal.

\begin{definition}
	A \textit{partition} of a set $A$ is a collection $\mathcal{D}$ of disjoint non empty subsets of $A$ whose union is $A$.
\end{definition}

So a collection of all equivalence classes form a partition of $A$. Conversely, given a partition of $A$, we can create an equivalence relation $x \sim y \iff x \in P_i, y \in P_i$ where $P_i$ is one of the disjoint subsets. We denote the set of all equivalence clss of an equivalence relation as $A/\sim$. Note the similarity to quotienting, and how the process is basically the same but with more generality.

\begin{definition}[Munkres]
A partial order of a set $A$ is a relation $\leq$ on $A$ which is reflexive, anti-symmetric ($x \leq y, y \leq x \implies x = y$) and transitive. A partial order is called a total order if every pair of two elements of $A$ are comparable.
\end{definition}
From now on, when we refer to ordered sets we are referring to totally ordered sets.
If $(A, \leq)$ is an ordered set, and $a < b$, then we define the open interval $(a, b) = \{x \in A\mid a < x < b\}$ where $<$ means the obvious non-equality thing. If $(a, b) = \varnothing$ then $b$ is called the immediate successor of $a$ and $a$ is called the immediate predecessor of $b$. We can similarly define other types of intervals, in fact it's done in Munkres chapter 2.

\begin{definition}
Suppose $(A, \leq_1)$ and $(B, \leq_2)$ are two ordered sets. Then we define the dictionary order relation on $A \times B$ by $(a, b) < (a', b')$ if $a <_1 a'$ or $a = a'$ and $b<_2 b$.
\end{definition}

\begin{definition}
If $A$ and $B$ are ordered sets we say that $A$ and $B$ have the same order type or are order isomorphic if there is a bijection $f: A \to B$ which is order preserving: $a < a' \implies f(a) <_2 f(a')$.
\end{definition}

$\mathbb{N}$ is not order isomorphic to $\mathbb{Z}$: note that $\mathbb{N}$ has a least element. Similarly, $\mathbb{Z}$ is not order isomorphic to $\mathbb{Q}$ because integers have immediate successors and predecessors and no rational number has that property.

\begin{definition}
	Let $(A, \leq)$ be an ordered set. $b \in A$ is a largest element of $A$ if $b \geq a$ for all $a \in A$. $b$ is an upper bound for $X$ if $b \geq x$ for all $x \in X$. We say that $b$ is a \textbf{least upper bound} if $b$ is an upper bound but for any upper bound $b'$ for x we have $b \leq b'$.
	Least elements, lower bounds, and greatest lower bounds can be defined analogously.
\end{definition}

Finally, Tamvakis discussed some axioms about ``infinity'', which he did not believe in. (He is a finitist.) 
\begin{definition}
	The \textbf{Axiom of Infinity} essentially asserts the existence of $\mathbb{N}$.
\end{definition}

Related to infinity, suppose that $\{A_i\}_{i \in I}$ is a family of sets, with $A_i neq \varnothing$ for all $i \in I$. We want to define the product 
\[\prod_{i \in I}A_i.\]
If I is finite, then WLOG $I = \{1, \dotsc, n\}$ and we can just define the product as the Cartesian Product. How do we define an infinite product? The idea is to think of the tuple $(a_1, \dots, a_n)$ as a function with values $a(1), \dotsc, a(n)$ where $a_i \in A_i$ for all $i$. 

\begin{definition}
	The product 
	\[\prod_{i \in I}A_i\] is the set of all functions \[a: I \to \bigcup_{i \in I}A_i\] such that $a(i) \in A_i$ for all $i \in I$. 
\end{definition}

The famous \textbf{Axiom of Choice} asserts that this product is non-empty. Such a function $a$ is called a \textit{choice function} for the family of sets.

\section{Week 2: More Set Theory Review}

\subsection{Counting and Cardinality}

\begin{definition}
A set $A$ is \textit{finite} if there is a bijection $f:A \to [n]$ ($[n] = \{1, 2, 3, \dotsc, n\}$), or if $A = \varnothing$.
If $A$ is not finite, then we call $A$ \textit{infinite}.
\end{definition}

One can check for themselves that finite unions and finite cartesian products are finite. 

\begin{definition}
A set $A$ is countably infinite if there is a bijection $f: \mathbb{N} \to A$. $A$ is called \textit{countable} if $A$ is finite or countably infinite.
\end{definition}

\begin{theorem}
A countable union $\bigcup_{n = 1}^{\infty}A_n$ of countable sets $A_n$ is countable. (Note that we can account for finite unions by setting $A_m = \varnothing$ for sufficiently large $m$.
\end{theorem}

\begin{proof}
Assume that the union is infinite.

If $A_j$ is countable we can write $A_j = \{a_{ij}\}_{i \geq 1}$. Otherwise, if $A_j$ is finite, then we can construct an infinite sequence by defining $A_n = a_{kj}$ for $n \geq k$.

We can list the elements of these sets as an infinite double array 
\[
\begin{matrix}
a_{11} & a_{12} & a_{13} & a_{14} & \dots \\
a_{21} & a_{22} & a_{23} & a_{24} & \dots \\
a_{31} & a_{32} & a_{33} & a_{34} & \dots \\
a_{41} & a_{42} & a_{43} & a_{44} & \dots \\
\vdots &\vdots &\vdots &\vdots & \ddots 
\end{matrix}
\]

If the index of the union is finite we can enumerate all the elements by going up and down the columns, omitting repeats. Otherwise we can enumerate the elements along the anti-diagonals, omitting repeats as needed. The end result exhibits a bijection between $\mathbb{N}$ and $\bigcup_{i = 1}^{\infty}A_n$.
\end{proof}

From this we easily get that the two sets $\mathbb{Z}$ and $\mathbb{Q}$ are countable sets.

\begin{theorem}
The finite product $A_1 \times \cdots \times A_n$ of countable sets is countable.
\end{theorem}

\begin{proof}
By an inductive argument it will suffice to show this for two sets $A \times B$. If we enumerate $A = \{a_j\}_{j \geq 1}$ as in the previous proof, we have that 
\[A \times B = \bigcap_{a \in A}\{(a, b) \mid b \in B\}\] which is a countable union of countable sets. Hence by the previous theorem $A \times B$ is countable.
\end{proof}

This does not extend to infinite products, as the following theorem shows.
\begin{definition}
If $A$ and $B$ are sets, define $A^B = \{f \mid f:B \to A\}$.
\end{definition}
\begin{theorem}
	There is no injection $I: \{0, 1\}^{\mathbb{N}} \to \mathbb{N}$.
\end{theorem}

\begin{proof}
Suppose such an $I$ exists. Then we can list the members of the domain in a sequence $S = \{\bold{x}_1, \bold{x}_2, \dots\}$ where 
\begin{align*}
\bold{x}_1 &= \{x_{11}, x_{12}, \dots\}\\
\bold{x}_2 &= \{x_{21}, x_{22}, \dots\}\\
\bold{x}_3 &= \{x_{31}, x_{32}, \dots\}\\
	\dots
\end{align*}

Consider the sequence $y = (y_1, \dots, y_n, \dots)$ where $y_j = 1 - x_{jj}$. This is not equal to any of the $\bold{x}_k$ for any $k$, a contradiction.
\end{proof}

As a corollary $\mathbb{R}$ is uncountable.

But of course, there are sets which are greater then $\mathbb{R}$ in cardinality!

\begin{theorem}
For any set $A$ there is no onto function $f:A \to \mathcal{P}(A)$.
\end{theorem}
\begin{proof}
	Consider any function $g: A \to \mathcal{P}(A)$ and WLOG we will assume that $A \neq \varnothing$. Consider the set 
	\[B = \{a \in A | a \not\in g(a)\} \subset A.\]
	We claim that $B$ is not in the image of $g$. For we have
	\[a_0 \in B \iff g(a_0) = B \iff a_0 \not\in B,\]
	a contradiction.
\end{proof}

As a corollary we have that $\mathcal{P}(\mathbb{N})$ is uncountable.
We can define an ``equivalence relation'' $\sim$ on sets by saying $A \sim B$ if and only if $A \simeq B$ (that is, there is a bijection between the two sets). We'd like to assign a ``cardinal number'' $o(A)$ to any set $A$, with the property that $o(A) = o(B) \iff A \simeq B$. A serious question.

\begin{question}
Is this ``equivalence relation'' even well defined? 
\end{question}

This is a question that hopefully a course like MATH446 would answer. For now we'll just say that $o(A)$ is a ``thing'', where $o(A) = o(B)$ if and only if $A \simeq B$, with the properties that $o(\mathbb{N}) = \aleph_0$ and $o(\mathbb{R}) = \mathfrak{c}$. 

\subsection{Well Orderings}

We might want to refine last lecture's equivalence relation to a partial order relation $\leq$ where $o(A) \leq o(B)$ if and only if there exists an injective function $f$ from $A$ to $B$. Showing that this is an equivalence relation in the sense that it satisfies the properties of an equivalence relation is difficult.

\begin{theorem}
Suppose $A$ and $B$ are sets, and $f: A\to B$ is injective and $g: B \to A$ is injective. Then $o(A) = o(B)$, ie a bijection exists between $A$ and $B$.
\end{theorem}
\begin{proof}
This is the \textit{Schroeder-Bernstein Theorem}. It is assigned as a problem to do for the homework.
\end{proof}

We will begin by developing some of the theory of well-ordered sets.

\begin{definition}
	A set $(A, \leq)$ with a total order relation is called well-ordered if every non-empty subset $B \in A$ has a smallest element. A well-ordered set is called an ordinal.
\end{definition}

For example, $\mathbb{Z}_{\geq 0} = \{0, 1, 2, \dots\}$ with the usual order is well-ordered. We might also create another well-ordered set in the following way: let $\omega$ be an abstract element, and consider $\mathbb{Z}_{\geq 0} \cup \{\omega\}$ with $\leq$ extending the order of the previous set by just setting $\omega$ to be the maximal element of the set. Clearly the two are not order-isomorphic. We will write $[0, \omega]$ for this well-ordered set, and $[0, \omega)$ for just $\mathbb{Z}_{\geq 0}$.

As another example of a well ordered set, consider $\{0, 1\} \times [0, \omega)$ in the dictionary order. It is clear that it is well ordered because we can explicitly construct the minimum element by using the well-ordering of $\mathbb{Z}_{\geq 0}$. We generalize this example below.

\begin{theorem}
Consider $[0, \omega) \times [0, \omega)$ in the dictionary order. We claim that this is a well-ordered set.
\end{theorem}
\begin{proof}
Suppose $A \subset [0, \omega) \times [0, \omega)$ is some non-empty subset. Let $A_1$ be the subset $[0, \omega)$ consisting of all the first coordinates of elements in $A$. Since its superset is well ordered there must be a least element of $A_1$ called $a_0$. Now we can define $B$ analogously as 
\[B = \{b \in [0,\omega) \mid (a_0, b) \in A\}.\]
By the same reasons as before $B$ has a smallest element $b_0$ and by construction $(a_0, b_0)$ is the minimal element of $A$.
\end{proof}

Now suppose $(X, \leq)$ is well-ordered. Then any non-maximal $x \in X$ has an immediate successor denoted $x^+$ or $x + 1$. This is clear if you consider that the set $\{y \in X \mid y > x\}$ has a least element $y_0$.

However, $x$ need not have an immediate predecessor, even if $x$ is not the minimal element.

\begin{definition}
	If $X$ is a well-ordered set, and $a \in X$, the \textit{section} of $X$ by $a$ is the set 
	\[S_a = \{x \in X \mid x < a\}.\] We also denote $S_a$ by $[0, a]$.
\end{definition}

It is an interesting property that $[0, \omega)$ is a countable and well-ordered set every section of which is finite. In fact this property uniquely determines the order type of $[0, \omega)$. As such we call $[0, \omega)$ the minimal countable well-ordered set.

The following theorem by Zermelo in 1904 reveals the astonishing fact:
\begin{theorem}
For any set $A$ there exists an order relation on $A$ that is a well-ordreing of $A$. In particular, there exists an uncountable well-ordred set.
\end{theorem}
The proof uses the Axiom of Choice.

\begin{proposition}
There exists a well ordered set having a largest element $\Omega$ such that $S_{\Omega}$ is uncountable, but every other section of $[0, \omega]$ is countable!
\end{proposition}

\section{Week 3: What's a Topology? How do we represent one?}

We might recall the definition of a sequence. How do we generalize this, possibly to other spaces without a metric? The key idea is to define ``open sets'' in any set $X$, and to use these to define open neighborhoods of points $x \in X$.

\begin{definition}
Let $X$ be a set. A topology on $X$ is a family of subsets $\tau$ of $X$ such that 
\begin{itemize}
	\item $\varnothing$ and $X$ are in $\tau$
	\item Arbitrary unions of members of $\tau$ lie in $\tau$
	\item Any finite intersection of members of $\tau$ lies in $\tau$.
\end{itemize}
Then we call the pair $(X, \tau)$ a $\textbf{topological space}$. The elements of $X$ are called \textit{points} and the elements $U$ of $\tau$ are called open sets of $(X, \tau)$. 
\end{definition}
\begin{definition}
A neighborhood (abbr. nbhd) of a point $x \in X$ is an open set $U$ containing $x$. The points of $U$ are said to be called \textbf{$U$-close} to $x$.
\end{definition}

\begin{example}
Let $X$ be any set and let $\tau = \{\varnothing, X\}$. This is the smallest possible topology on $X$, called the indiscrete topology.

Let $X$ be any set, and let $\tau = \mathcal{P}(X)$. This is called the discrete topology on $X$. That is, all subsets of $X$ are open.

Let $X = \{a,b\}$ and let $\tau = \{\varnothing, \{a\}, \{a, b\}$. Then $(X, \tau)$ is called the \textbf{Sierpinski space}.

Let $X$ be any set, and let $\tau$ be the family of subsets $U$ of $X$ such that $X \setminus U$ is finite or all of $X$. Then $\tau$ is a topology called the \textbf{finite complement topology}.

Let $d$ be a distance function on $X$. Then a pair $(X, d)$ of a set $X$ with a metric $d$ on $X$ is called a metric space. In any metric space $(X, d)$ we can define for any $\varepsilon > 0$ the $\varepsilon$-ball 
\[B(x, \varepsilon) = \{y \in X | d(x, y) < \varepsilon\}.\]
Then we can define a topology $\tau$ on $X$ called the metric space topology by $U \subset X$ is open if for all $x \in U$ there exists $\varepsilon > 0$ such that $B(x, \varepsilon) \subset U$. 
\end{example}

\begin{definition}
Let $\tau$ and $\tau'$ be two topologies on a set $X$. If $\tau' \supset \tau$ then we say that $\tau'$ is finer than $\tau$, and $\tau$ is coarser than $\tau'$. We say that $\tau$ is comparable to $\tau'$ if they are ordered with respect to subset inclusion.
\end{definition}

\subsection{Bases}

\begin{definition}
Let $(X, \tau)$ be a topological space. A subset $\mathcal{B} \subset \tau$ is called a basis for $\tau$ if every $U \in \tau$ is a union of some members of $\mathcal{B}$. 
\end{definition}

For example, by definition the family of $\varepsilon$-balls in a metric space is a basis generating that topology.

Here is another way to phrase the definition:
\begin{definition}
A collection $\mathcal{C} \subset \tau$ is a basis of $\tau$ if and only if for all $U \in \tau$ and all $x \in U$ there exists $C \in \mathcal{C}$ such that $x \in C \subset U$.
\end{definition}

\begin{question}
Suppose $B \subset \mathcal{P}(X)$. Is $B$ a basis for a topology of $X$?
\end{question}

The answer is no, not necessarily.

\begin{theorem}
Let $\mathcal{B}$ be a family of subsets of $X$ that satisfies the following condition: if $B, C \in \mathcal{B}$ and $x \in B \cap C$, then there exists $D \in \mathcal{B}$ such that $x \in D \subset B \cap C$.
Then $\mathcal{B}$ is a basis for some topology on $X$ containing $\varnothing$ and $X$.
\end{theorem}

\begin{proof}
By definition, $\varnothing$ and $X$ are in $\tau$. Suppose $\{U_i\}_{i \in I}$ is a family of elements of $\tau_{\mathcal{B}}$. Each $U_i$ is either $\varnothing$, $X$, or a union $\bigcup_{j \in I_i}B_j = U_i$, so the union of the $U_i$'s is as well.

Suppose $U_1, U_2 \in \tau_{\mathcal{B}}$. We will show that their intersection is as well. Take $x \in U_1 \cap U_2$. Since $x \in U_i$ there exists $B_1 \in \mathcal{B}$ and $B_2 \in \mathcal{B}$ such that $x \in B_1$, $x \in B_2$, so $x \in B_1 \cap B_2$. Then we know there is a $B_3$ such that $x \in B_3 \subset B_1 \cap B_2 \subset U_1 \cap U_2$, so we are done. Any arbitrary finite union follows from induction.
\end{proof}

\begin{proposition}
Let $\mathcal{B}$ and $\mathcal{B}'$ be bases for the topologies $\tau$ and $\tau'$ on $X$, respectively. Then the following are equivalent:
\begin{itemize}
	\item $\tau'$ is finer than $\tau$.
	\item For all $x \in X$ and every $B \in \mathcal{B}$ there exists $B' \in \mathcal{B}'$ such that $x \in B' \subset B$.
\end{itemize}
\end{proposition}
\begin{proof}
Let $U \in \tau$. Then $U$ is a union of elements of $\mathcal{B}$ which implies that $U$ is a union of elements of $\mathcal{B}'$. So $U \in \tau'$. 

Conversely, given an $x \in \mathcal{B}$, we have that $B \in \mathcal{B} \in \tau \subset \tau'$. So $B \in \tau'$ by definition, so we have a $B'$ (namely $B$) such that $x \in B' \subset B$.
\end{proof}

\begin{definition}
We define $3$ topologies on $\mathbb{R}$:
\begin{enumerate}
	\item Consider the basis $\{(a,b) | a < b\}$. This basis generates the usual topology on $\mathbb{R}$.
	\item Consider instead the basis $\{[a,b) \mid a < b\}$. The topology $\tau'$ generated by this basis is called the lower limit topology.
	\item Let $K = \{1/n \mid n \in \mathbb{N}\}$. Now consider the basis $\{(a, b) \mid a < b\} \cup \{(a, b) - K \mid a < b\}$. We will call this the $K$-topology on $\mathbb{R}$.
\end{enumerate}
We will let $\mathbb{R}$, $\mathbb{R}_{\ell}$, and $\mathbb{R}_K$ denote these topological spaces, respectively.
\end{definition}

\begin{proposition}
The topologies $\mathbb{R}_{\ell}$ and $\mathbb{R}_K$ are strictly finer than the standard topology on $\mathbb{R}$. However, they are not comparable with each other.
\end{proposition}

\begin{definition}
Let $\sigma = \{A_i\}_{i \in I}$ be any family of subsets of $X$. Then there exists a unique smallest topology $\tau(\sigma)$ which contains $\sigma$. The family $\tau(\sigma)$ consists of $\varnothing$, $X$, all finite intersections of the $A_i$, and all arbitrary unions of these finite intersections. $\sigma$ is called a subbasis for $\tau(\sigma)$ and $\tau(\sigma)$ is said to be generated by $\sigma$.
\end{definition}

By construction $\tau(\sigma) \supset \sigma$. If $\tau'$ is another topology on $X$ containing $\sigma$ then $\tau(\sigma) \subset \tau'$. We note that $\tau(\sigma)$ is as described because $\sigma \in \tau(\sigma)$.

Since $\bigcup$ distributes over $\bigcap$, the latter collection, together with $\varnothing$ and $X$ forms a topology.

\subsection{The Order Topology}

Let $(X, \leq)$ be a totally ordered set. Let $\mathcal{B}$ be the collection of open intervals $(a, b)$, or $[a_0,b)$ or $(a, b_0]$ where $a_0$ and $b_0$ are the minimal and maximal element of $X$, if they exist.

\subsection{The Product Topology}

Given $(X, \tau_X)$ and $(Y, \tau_Y)$, two topological spaces, the product topology on $X \times Y$ denoted $\tau_X \times \tau_Y$ is the topology having basis $\mathcal{B}$ of all sets of the form
\[\mathcal{B} = \{U \times V \mid U \in \tau_X, V \in \tau_Y\}.\]

This forms a basis, because we have
\[(U_1 \times V_1) \cap (U_2 \times V_2) = (U_1 \cap U_2) \times (V_1 \cap V_2),\] so we can find a third basis element easily in their intersection.

\begin{proposition}
If $(X, \tau_X)$ and $(Y, \tau_Y)$ are $2$ topological spaces with bases $\mathcal{B}$ and $\mathcal{B}'$, respecitvely, then the family $\mathcal{C} = \{\mathcal{B} \times \mathcal{B}'\}$ is a topology for $X \times Y$. 
\end{proposition}

Note that the two statements in the definition are different: the former deals with elements in the topology, and the latter deals with only basis elements in the topology. 

\begin{proof}
Let $W \in X \times Y$ be open, and $(x, y) \in W$. By definition there exists $U \in \tau_X$ and $V \in \tau_Y$ such that $(x, y) \in U \times V \subset W$. Then we choose $B \in \mathcal{B}$ and $B' \in \mathcal{B}'$ such that $(x, y) \in B \times B' \subset U \times V \subset W$.
\end{proof}

\begin{definition}
The maps $\pi_1:X \times Y \to X$ where $(x, y) \mapsto x$ and $\pi_2: X \times Y \to Y$ where $(x, y) \mapsto y$ are called the two projections of $X \times Y$ to the first and second factor, respectively.
\end{definition}

Note that $\pi_1^{-1}(U) = U \times Y$ and $\pi_2^{-1}(V) = X \times V$. 

\begin{proposition}
The family $\sigma$ consisting of inverse images of $\pi_1$ of open sets $U$ and inverse images of $\pi_2$ of open sets $V$ is a subbasis for $X \times Y$.
\end{proposition}

\begin{proof}
Observe that $\pi_1^{-1}(U) \cap \pi_2^{-1}(V) = U \cap V$. Well, these are just the basis elements which generate $X \times Y$. So the claim is true.
\end{proof}

\subsection{The Subspace Topology}

Sometimes it is useful to define topologies on a subset of a topological space, regardless of whether that set itself is closed or open.

\begin{definition}
Let $(X, \tau)$ be a topological space. If $Y \subset X$ is any subset then the family $\tau_Y = \{U \cap Y \mid U \in \tau\}$ is a topology on $Y$ called the subspace topology.
\end{definition}

It is pretty easy (DeMorgan) to show that this endows $Y$ with a topology. Another thing which is simple to prove is the following:

\begin{proposition}
If $\mathcal{B}$ is a basis for $\tau$, then $\mathcal{B}_Y = \{B \cap Y \mid B \in \mathcal{B}\}$ is a basis for $\tau_Y$.
\end{proposition}

\begin{proof}
Suppose $x \in (B_1 \cap Y) \cap (B_2 \cap Y)$. By the associativity and commutativity of set intersection this is equivalent to $x \in (B_1 \cap B_2) \cap Y$. Since $B_1$ and $B_2$ are basis elements there exists an element $B_3$ such that $x \in B_3 \subset B_1 \cap B_2$. It follows that $x \in B_3 \cap Y \subset (B_1 \cap B_2) \cap Y$ and we are done.
\end{proof}

\begin{definition}
If $Y \subset X$ is a subspace of $X$, then we say a set $U$ is open in $Y$ if $U \in \tau_Y$, and $U$ is open in $X$ if $U \in \tau$.
\end{definition}

\begin{proposition}
Let $Y \in X$ be a subspace of $X$. If $U$ is open in $Y$ and $Y$ is open in $X$ then $U$ is open in $X$.
\end{proposition}
\begin{proof}
$U$ = $V \cap Y$ for some open set $V$ in $X$. But if $Y$ is open, it follows that $V \cap Y$ is open as well.
\end{proof}

\begin{theorem}
If $A \subset X$ and $B \subset Y$ are subspaces, then the product topology of $A \times B$ is the same as the subspace topology which $A \times B$ inherits from $X \times Y$.
\end{theorem}
\begin{proof}
This follows from the fact that $(U \times V) \cap (A \times B) = (U \cap A) \times (V \cap B)$. But $U \cap A$ is an open set in $A$ and $(V \cap B$ is a general open set in $B$. Hence there is a bijective correspondance between the two topologies, hence they are equal.
\end{proof}

\begin{example}
Suppose $(X, \leq)$ is an ordered set with the order topology, and $Y \subset X$. Then $Y$ inherits an order from $X$. So we get an order topology on $(Y, \leq)$. This topology need not be the same as the subspace topology. For example, consider the interval $I = [0,1]$, and the set $I \times I$ with the dictionary order. %todo
\end{example}

\section{Week 4: Closed sets and Closure}

\begin{definition}
Let $X$ be a topological space. We have that $F \subset X$ is closed if $F^c = X - F$ is open.
\end{definition}

\begin{example}
If $X = [0, 1) \cup \{2\}$ with the subspace topology, then $[0, 1)$ and $\{2\}$ are both open and closed in $X$. Later we will see that this means that $X$ is not connected.
\end{example}

\begin{theorem}
In any topological space $X$, the following are true:
\begin{itemize}
	\item $\varnothing$ and $X$ are closed
	\item Arbitrary intersections of closed sets are closed.
	\item Finite unions of closed sets are closed.
\end{itemize}
\end{theorem}
\begin{proof}
The first point is clear. The second and third points follow from DeMorgan's Laws.
\end{proof}

One may notice that the definition of a closed set is not \textit{intrinsic}, that is, it is defined as sort of a dual to open sets. The following definitions try to make closed sets more intrinsic.

\begin{definition}
	Let $A \subset X$ be any subset. A point $x \in X$ is called adherent to $A$ if any neighborhood $U$ of $X$ contains at least one point of $A$, possibly $x$ itself.
\end{definition}
\begin{definition}
The \textbf{closure} of $A$ is the set 
\[\overline{A} = \{x \in X \mid \forall (\text{neighborhoods $U$ of $x$}), U \cap A \neq \varnothing\}.\]
\end{definition}

\begin{example}
In $\mathbb{R}$, let $A = \{(x, \sin(1/x)) \mid 0 < x \leq 1\}$. Then $\overline{A} = A \cup (\{0\} \times [-1, 1])$.
\end{example}

As a remark, if the topology on $X$ is given by a basis $\mathcal{B}$, then $x \in \overline{A}$ is true if and only if every $B \in \mathcal{B}$ with $x \in B$ intersects $A$ non-trivially (ie not $\varnothing$).
\end{document}
