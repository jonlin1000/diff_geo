\documentclass[12pt]{article}
\usepackage[margin=1in]{geometry} 
\usepackage{amsmath}
\usepackage{tcolorbox}
\usepackage{amssymb}
\usepackage{amsthm}
\usepackage{lastpage}
\usepackage{fancyhdr}
\usepackage{accents}
\pagestyle{fancy}
\setlength{\headheight}{40pt}


\newenvironment{solution}
  {\renewcommand\qedsymbol{$\blacksquare$}
  \begin{proof}[Solution]}
  {\end{proof}}
\renewcommand\qedsymbol{$\blacksquare$}

\newcommand{\ubar}[1]{\underaccent{\bar}{#1}}

\theoremstyle{plain}
\newtheorem{proposition}{Proposition}
\newtheorem{theorem}{Theorem}
\newtheorem{lemma}{Lemma}
\newtheorem{corollary}{Corollary}
\theoremstyle{definition}
\newtheorem{definition}{Definition}
\newtheorem{example}{Example}

\newcommand{\ord}{\operatorname{ord}}

\lhead{Commutative Algebra}
\rhead{Jonathan Lin \\ \today} 
\cfoot{\thepage\ of \pageref{LastPage}}


\begin{document}
\section{Preliminaries}

First we will recall the definition of a ring:
\begin{definition}
A ring is a set $R$ along with two operations $+$ and $\cdot$ with the following properties:
\begin{itemize}
\item $(R, +)$ is an abelian group.
\item $\cdot$ is distributive over $+$, so that $a\cdot(b + c) = a \cdot b + a \cdot c$ and $(a + b) \cdot c = a \cdot c + b \cdot c$ for all $a, b, c \in R$.
\item $\cdot$ is associative.
\end{itemize}
If $\cdot$ is commutative we call $R$ a \textit{commutative} ring. If $R$ has a multiplicative identity $1$ (that is, an element such that $1\cdot r = r\cdot 1 = r$ for all $r \in R$), we say that $R$ is a \textit{unital} ring.
\end{definition}

From now on we will assume that our rings are commutative (and unital?) unless otherwise stated.

\begin{definition}
Suppose $R$ is a ring. Then $I \subset R$ is an \textbf{ideal} if $RI \subseteq I$ or $IR \subseteq I$ (where the terms $RI$ and $IR$ are the usual notation). $I$ is called \textit{finitely generated} if 
\[I = \langle a_1, \dots, a_n \rangle = \{a_1r_1 + \cdots + a_nr_n \mid r_i \in R\}.\]
\end{definition}

Here are some basic examples of rings.

\begin{example}
The ring of integers $\mathbb{Z}$ with usual integer addition and multiplication is a commutative ring with unity. Same with $\mathbb{Q}$, $\mathbb{R}$, and $\mathbb{C}$.
\end{example}
\begin{example}
Let $R$ be a ring. Then the polynomial ring $R[x]$ with coefficients in $R$ with the usual multiplication and addition is a ring.
\end{example}
\begin{example}
The sets $\mathbb{Z}/n\mathbb{Z}$ can be made into rings with addition and multiplication modulo $n$.
\end{example}

\section{Hilbert's Basis Theorem}

\begin{definition}
Let $R$ be a ring. Then $R$ is \textbf{Noetherian} (after Emmy Noether) if every ideal of $R$ is finitely generated.
\end{definition}
\begin{proposition}
$R$ is Noetherian if and only if it satisfies the \textit{ascending chain condition}: For every sequence of ideals
\[I_1 \subset I_2 \subset I_3 \subset \cdots\] there exists some $n \in \mathbb{N}$ such that 
\[I_n = I_{n + 1} = I_{n + 2} = \cdots.\]
\end{proposition}

\begin{example}
Any PID is Noetherian, and so is any field $F$.
\end{example}
\begin{example}
Consider the field $R = F[x_1, x_2, \dots]$ of infinitely many indeterminates where $F$ is a field. Then we have an increasing sequence of ideals
\[\langle x_1 \rangle \subset \langle x_1, x_2 \rangle \subset \langle x_1, x_2, x_3 \rangle \subset \cdots\]
where none of the ideals in the chain are equal to one another.
\end{example}

We are ready to state Hilbert's Basis Theorem.
\begin{theorem}
If $R$ is a Noetherian ring, then $R[x]$ is a Noetherian Ring. It also follows that $R[x_1, \dots, x_n]$ is a Noetherian ring by induction.
\end{theorem}
\begin{proof}
Suppose that $I$ is an ideal in $R[x]$. Denote $LC(I)$ to be the leading coefficients of elements in $I$. It is straightforward to check that $LC(I)$ is an ideal in $R$. Since $R$ is Noetherian, $LC(I)$ is finitely generated. This means
\[LC(I) = \langle a_1, \dots, a_n \rangle.\]
By definition, $a_i$ is the leading coefficient of some polynomial $f_i(x) = a_ix^n_i + \cdots$. Let $N$ be the maximum degree of the $f_i$. Now define $LC(I_d)$ to be the leading coefficients of elements in $I$ that have degree $d$ (and $0$). It is easy to show that $LC(I_d)$ is also an ideal of $R$. So
\[LC(I_d) = \langle b_{d,1}, \dots, b_{d, n_d} \rangle\] and we can let $f_{d, i}(x)$ be the polynomials with the $b_{d, i}$ as leading coefficient.

We claim that 
\[I = \langle f_i, f_{d, j} \rangle\] where $i$ rangles from $1$ to $n$, $d$ ranges from $1$ to $n$, and $j$ ranges from $1$ to $n_d$ (depending on $d$). This will show that $I$ is finitely generated.

Suppose not. Then take a polynomial $f \in \langle f_i, f_{d, j} \rangle$ of minimal degree. If $\deg{f} \geq N$, then we can find polynomials $q_i(x)$ such that $\sum_{i = 1}^nq_if_i(x)$ has the same leading term as $f$. Subtracting we use the minimal degree assumption and we deduce that $f$ is in $\langle f_i \rangle$.
Otherwise, if $\deg{f} = d < n$ then we can do the same thing but with the $f_{d, i}$ instead. This shows that $I$ is in fact $\langle f_i, f_{d, j}\rangle$ as desired.
\end{proof}

\section{Monomial Ordering}
\end{document}